
\begin{definition}
    Let $\mu:\Sigma^*\rightarrow M$ be a monoid morphism and let $w$ be in $ \Sigma^*$.
    A $k$-factorization of $w$ in the morphism $\mu$ is given as a tuple of words $(w_0,x_{1},y_1,z_1,w_1,\ldots,x_k,y_k,z_k, w_k)$ verifying:
    \begin{itemize}   
        \item  $w=w_0x_1y_1z_1w_1\cdots x_ky_kz_kw_k$
        \item for all $i\in [1,k]$, $\mu(x_i)=\mu(y_i)=\mu(z_i)=\mu(x_ix_i)$
    \end{itemize}
    We say that such a factorization is \emph{according to} $(m_0,e_1,m_1,\ldots,e_k,m_k)$ if for all $i\in [0,k]$, $\mu(w_i)=m_i$ and for all $i\in [1,k]$, $\mu(x_{i})=e_i$.
\end {definition}




\begin{comment}
\begin{lemma}
Let $\Tt=\tuple{T_k,\ldots,T_1}$ be a $k$-pebble transducer over input alphabet $\Sigma$.
One can obtain a new $k$-pebble transducer $\Tt'=\tuple{T_k',\ldots,T_1'}$ over $\Sigma'=\Sigma\times L $ such that:
\begin{itemize}
\item for any $w\in \Sigma'^*$, $\Tt'(w)=\Tt\circ\pi_{\Sigma}(w)$
\item For any $\bigcup_{j<k} Q_j'$-producing triple $(w_1,w_2,w_3)$, we have $\Tt'(w_1w_2^kw_3)=\Theta(|w_2|^k)$
\end{itemize}

\end{lemma}
\end{comment}

\begin{lemma} 
    \label{lem:bounded}
    Let $\tuple{T_1,\ldots,T_{k+1}}$ be a pebble transducer realizing a function $f$, such that $T_{k+1}$ is bounded in $\set{k}$. Then $f$ can be realized by a $k$-pebble transducer.
\end{lemma}     

\begin{lemma}[Named Lemma]\label{lem:name}
    Let $\Tt=\tuple{T_1,\ldots,T_k}$ be a pebble transducer over input alphabet $\Sigma$ realizing a function $f$.
    There exists a morphism in a finite monoid $\mu:(\Sigma_{\vdash\dashv})^*\rightarrow M$ and a set $P\subseteq M^{2k+1}$ such that for all $p\in \NN$:
    \begin{itemize}
    \item For any $w\in \Sigma^*$ with a $k$ factorization 
    $(w_0,x_{1},y_1,z_1,w_1,\ldots,x_k,y_k,z_k, w_k)$ according to an element of $P$, $|f(w_1x_1y_{1}^nz_1w_1\cdots x_ky_k^nz_kw_k)|=\Theta(n^k)$.
    \item $f$ restricted to words without $k$ factorization according to any element of $P$ can be realized by a $k{-}1$-pebble transducer.
    \end{itemize}
    
\end{lemma}



\begin{proof}
    This is shown by induction on $k$.
    For $k=1$, it is a consequence of the proof of Theorem~\ref{thm:linear}.

    We assume that the lemma holds for $k$, let us show that it holds for $k+1$.
    Let $\Tt=\tuple{T_1,\ldots,T_k,T_{k+1}}$ be a pebble transducer realizing a function $f:\Sigma^*\rightarrow\Gamma^*$, and let $\Tt_k=\tuple{T_1,\ldots,T_k}$. Let $f_k:\Sigma(Q_{k+1})^*\rightarrow \Gamma^*$ be the function realized by $\Tt_k$.

    Let us apply the induction assumption to $\Tt_k$, and let $\mu:(\Sigma_{k,\vdash\dashv})^*\rightarrow M$ and $P$ be given as in the lemma, with $\Sigma_k=\Sigma(Q_{k+1})$. For any $p$, let $\Ss_p$ be a $k{-}1$-pebble transducer realizing the function $f_k$ restricted to words without any $k$ factorization according to elements of $P$, and let $g_p$ denote the function it realizes.

    The main idea of the proof is to modify the transducer $T_{k+1}$ into a new transducer which only outputs $k$ when it is \emph{absolutely necessary}, \textit{i.e.}~when the word can be factorized in such a way that, by pumping idempotents, one can obtain an output in $\Theta(n^k)$.
    Otherwise, we have according to the lemma that we can outsource the computation to a transducer with only $k-1$ pebbles.

    Let us define a new transducer $R_{k+1}$ which behaves as $T_{k}$, except that at each step where it should output the letter $k$, it checks, using some regular look-around if the word has a $(k,p)$ factorization according to an element of $P$. If yes then it outputs $k$ normally, calling $\Tt_k$, otherwise it calls $\Ss_p$ instead.

    The look-around is implemented by a rational function $\ell$ which labels each position by additional information.
    Let $L=Q_{k+1}\rightarrow \set{\Ss_p,\Tt_k}$ be the labelling alphabet, then $\ell:(\Sigma)^*\rightarrow (\Sigma\times L)^*$ is defined below:
    Let $w\in (\Sigma)^*$, the word $z=\ell(w)$ has the same size as $w$ and $z[i]=(w[i],h)$ with $h(q)=\Tt_k$ if and only if the word obtained by replacing $w[i]$ with $w[i](q)$ has a $k$ factorization according to an element of $P$.
    Let $\Lambda=\Sigma\times L$ in the following.


    \begin{claim}\label{claim:pump}
        Let $u=\ell(v)$ be in $ \Lambda^*$.
        Let us consider $(w_0,x_1,y_1,z_1,w_1)$ be a $1$ factorization of $v$.
        Let $u_n=\ell(w_0x_1y_1^nz_1w_1)$, then there exists $\alpha,\beta,\gamma\in \Lambda^*$ such that $u_n=\alpha\beta^n\gamma$, for all $n\in \NN$.
    \end{claim}
\begin{proof}[Claim~\ref{claim:pump}]
    Shown using Lemma~\ref{lem:2idem}.
\end{proof}
From the above claim, we have that we can pump an idempotent of $M$ without affecting the labelling.
This will not be sufficient however to obtain our final result. We also want to be able to pump an idempotent without changing the \emph{shape} of a run of $R_{k+1}$, that is its transitions without regard to the outputs.
For this we consider the transition morphism $\mu_{\mathbf 0}:(\Sigma_{\vdash,\dashv})^*$ of $T_{k+1}$. Indeed, the shape of a run of $R_{k+1}$ over a word $\ell(v)$ is the same as the shape of a run of $T_{k+1}$ over $v$.
Let $\nu=\mu\times \mu_{\mathbf 0}$ denote this product.

Let us consider the $\set{k}$-linear triplets of $R_{k+1}$.
We need to show two things, that the function over words whithout $\set{k}$-linear triplets can be done by a $k{-}1$-pebble transducer, and that otherwise we can find a $k$-factorization which yields $\Oo(n^k)$ output by pumping idempotents.
The first can be obtained from Lemma~\ref{lem:bounded}. The second point is shown using Claim~\ref{claim:pump}.
\end{proof}

\begin{theorem}
    Given a pebble transducer $\Tt$ realizing a function $f$ and a natural number $k$, one can decide if $f$ can be realized by a $k$-pebble transducer. In particular, one can decide if a polyregular function is regular.
\end{theorem}

\begin{corollary}
    A word-to-word function can be defined by an \mso interpretation of dimension $k$ if and only if it can be realized by a $k$-pebble transducer.
\end{corollary}